\chapter{Introduction}

\label{chap:intro}
Introduction to the introduction.
\section{Physics of Magnetised Plasmas}
Plasma is ionised gas, and thus listens to magnetic fields because it is charged.
\textit{TBD. Look in textbook, maybe write this last? All the physics you need to know to understand the next bit.}
Relevant equations (Maxwell etc) that will lead you to determining force balance equation, and also magnetic curvature force i.e. how jxb becomes pressure and curvature, and the navier-stokes one about pressure gradient and Fcent? i.e. conservation of momentum? That's basically already force balance.

\subsection{Plasma Waves}
Alfven, magnetosonic. (Do I need this section?)
\subsection{The Frozen-in Field Theorem}
Importance of beta in determining whether plasma follows Bfield or Bfield follows plasma. For  ideal MHD.
Non-ideal MHD and recconection.

\section{The Solar Wind} \label{intro:sec:solarwind}
The corona is the uppermost atmospheric layer of the Sun, and is composed mainly of ionised hydrogen gas (i.e. protons and electrons), with approximately $4\%$ ionised helium \citep{robbins1970}. This layer is heated from below by fusion processes in the deep interior, and other processes not yet fully understood, to extremely high temperatures of $\sim10^6\si{K}$ \citep{warren2009}. Above, the corona is surrounded by relatively empty space, and thus experiences a large thermal pressure gradient force directed radially outwards. This means that a significant fraction of the coronal plasma is energetic enough to escape the Sun's strong gravitational field, and thus streams radially outwards from the Sun through space at high speeds. This is known as the solar wind.

The properties of the solar wind vary on many timescales, however it is still useful to consider the typical properties. The typical solar wind speed is around $\SI{450}{kms^{-1}}$ throughout the solar system, well above the local speed of sound. The particle number density falls approximately as $r^{-2}$, in line with conservation of flux through an expanding spherical surface, from around $\SI{7}{cm^{-3}}$ at the orbit of Earth (\SI{1}{AU}) to \SI{0.07}{cm^{-3}} at the orbit of Saturn (\SI{9.5}{AU}) \citep{bagenal2014} (\SI{1}{AU} = \SI{1.496e8}{km}). Near the solar surface, the complex magnetic field of the Sun dominates the plasma dynamics, in line with the frozen-in field theorem, forming phenomena such as coronal loops. However further away from the Sun, beyond a few tens of solar radii, the magnetic field strength decreases with radial distance such that the remnant magnetic field becomes frozen-in to the solar wind plasma, and is then carried outwards into space, where it is known as the interplanetary magnetic field (IMF) \citep{russell2016}. The radial outflow of the solar wind, combined with the ${\sim}25$ day rotation period of the Sun, produce a spiral-like distribution of solar wind plasma in the Sun's equatorial plane which extends throughout the solar system, known as the `Parker Spiral' \citep{parker1958}. This influences the typical orientation of the interplanetary magnetic field observed at each planet, from approximately radial at the orbit of Mercury, to approximately perpendicular to the radial flow at the orbit of Saturn. Figure~\ref{intro:fig:parkerspiral} shows a diagram of this phenomenon. This has consequences for the interaction between the solar wind and the planet, as discussed in future sections.

\begin{figure}
\centering
\noindent\includegraphics[width=0.6\textwidth]{intro/parkerspiral.pdf}
\caption[Diagram of the Parker Spiral throughout the solar system.]{Diagram showing how the Parker Spiral affects the orientation of the IMF (interplanetary magnetic field) in the equatorial plane throughout the solar system. The orbits of different planets are shown by the coloured dotted lines.}
\label{intro:fig:parkerspiral}
\end{figure}

On top of this rotating structure, the properties of the solar wind can vary on timescales ranging from minutes to years. At the shorter end of the spectrum, coronal mass ejections (CMEs) are dynamic ejections of dense coronal material that occur due to reconfiguration of the coronal magnetic field. These typically form over days and may reach speeds as high as several thousand $\si{kms^{-1}}$ as they accelerate through the inner solar system. They are generally observed to be denser and have an enhanced magnetic field strength compared to the ambient solar wind \citep{russell2016}. At the opposite end of the spectrum, it is well known that the Sun exhibits periodic behaviour with an approximately 11 year cycle, known as the solar cycle. This cycle of alternatively high and low solar activity can be tracked well by the number of sunspots that appear on the solar surface, and is correlated with solar wind properties such as solar irradiance, magnetic field strength, and flare and CME incidences \citep{hathaway2015}. The magnetic and plasma environment of the solar system is therefore very variable.

\section{Planetary Magnetospheres}
\subsection{Structure of a Magnetosphere}
A magnetosphere is a bubble of magnetised plasma that surrounds a planet with a significant internal magnetic field, and forms due to the interaction between this magnetic field and the solar wind. Mercury, Earth, and the outer giant planets all have approximately dipolar internal magnetic fields, generated by convective flow of electrically conducting fluid in the planets' deep interiors \citep{kivelson2014book}. Thus they all have stable magnetospheres.

The magnetopause is the surface that separates the internal planetary plasma of the magnetosphere from the external shocked solar wind plasma of the magnetosheath. Upstream of the magnetosheath, the bow shock separates the shocked and un-shocked solar wind plasma. In a steady state system, the shape and size of the magnetopause is determined by pressure balance across the boundary, between the internal magnetic and plasma pressures, and the external solar wind pressure. This in turn influences the configuration of the magnetosphere internally. Therefore, the structure of the magnetosphere varies significantly between different planets, where both the internal and external conditions are different. However, many features are broadly common to all planetary magnetospheres in some form, and Figure~\ref{intro:fig:magnetosphere} shows a diagram specifically of Earth's magnetosphere, with these main features labelled. Note that at Jupiter and Saturn, the internal planetary magnetic field is oppositely oriented such that the magnetic fields and current systems are all in the opposite direction.

\begin{figure}
\centering
\noindent\includegraphics[width=0.8\textwidth]{intro/magnetospherediagram.jpg}
\caption[Diagram of Earth's magnetosphere.]{Diagram of the main features of Earth's magnetosphere from \citet{russell2016}, reproduced with permission.}
\label{intro:fig:magnetosphere}
\end{figure}

It can be seen that the magnetosphere is approximately dipolar in configuration on the dayside (i.e. the side facing the Sun), with a more extended `tail' on the nightside (i.e. the anti-sunward side), where the magnetic field lines stretch radially outwards and become approximately parallel to the solar wind direction. This tail can extend to tens or even hundreds of planetary radii downstream of the planet; Jupiter's magnetotail has been observed to extend as far as the orbit of Saturn \citep{scarf1981}. Figure~\ref{intro:fig:magnetosphere} shows an azimuthal ring current system which orbits the planet near the equatorial plane, extending on the nightside into a current sheet, which separates the oppositely directed magnetic field lines in the northern and southern `lobes' of the magnetosphere. Currents also flow on the magnetopause and magnetotail surfaces as shown, separating the magnetic fields of the magnetosphere and the magnetosheath/solar wind (i.e. the IMF). At the polar cusps, the magnetosphere is said to be `open' to the solar wind, as in these regions the internal planetary dipole magnetic field structure allows solar wind particles to access the magnetosphere. This is in contrast to the `closed' regions of the magnetosphere, where it is difficult for solar wind particles to penetrate.

\subsection{Comparative Magnetospheres}\label{intro:sec:comparativemagnetospheres}
A comparison of the magnetospheres of different planet systems is shown in Figure~\ref{intro:fig:magnetospherecomparison}. The most striking difference is in the overall size of the magnetospheres, and this is mainly due to the twin influences of the external solar wind conditions at each planet, and internal magnetic pressure associated with the planetary magnetic field. Table~\ref{intro:table:magnetospherecomparison} provides some typical parameters for the planets Earth, Jupiter and Saturn that help illustrate this. For example, the Jovian planetary dipole magnetic moment is some 20,000 times greater than that of Earth, with correspondingly higher magnetic field strength. As magnetic pressure scales as the square of magnetic field strength, this therefore means a much greater magnetic pressure inside the magnetosphere. Meanwhile, as discussed in Section~\ref{intro:sec:solarwind}, the mass density of the solar wind $\rho_\mathrm{m}$ falls approximately as $r^{-2}$ while the velocity $u$ remains approximately constant with radial distance. As the solar wind dynamic pressure scales with both such that $D_\mathrm{P} = \rho_\mathrm{m}u^2$, the external solar wind dynamic pressure is therefore much lower for the planets in the outer solar system, falling as $r^{-2}$ .

\begin{figure}
\centering
\noindent\includegraphics[width=1\textwidth]{intro/magnetospherecomparison.jpg}
\caption[Diagram of Mercury, Earth, Jupiter and Saturn magnetospheres.]{Diagram comparing the relative sizes and shapes of the magnetospheres of Mercury, Earth, Jupiter and Saturn, from Fran Bagenal and Steve Bartlett at LASP.}
\label{intro:fig:magnetospherecomparison}
\end{figure}
\begin{table}
\caption[Comparison of typical magnetospheric parameters for Earth, Jupiter and Saturn.]{Comparison of typical magnetospheric parameters for Earth, Jupiter and Saturn, adapted from \citet{bagenal2014} and references therein.}\label{intro:table:magnetospherecomparison}
\centering
\begin{tabular}{l c c  c}
\hline
  																															& Earth						& Jupiter			& Saturn  \\
\hline
Planet radius $R_\mathrm{P}$ ($\si{km}$)															& 6,371					&	71,492			&	60,268 \\
Distance from Sun ($\si{AU}$)																			&	1							&	5.2				& 9.5		\\
Solar wind number density ($\si{cm^{-3}}$)														& 7							&	0.2				&	0.07		\\
Spin period ($\si{hr}$)																						&	24						& 	9.9				&10.6		\\
Magnetic moment ($\si{M_\mathrm{EARTH}}$)													&	1							&	20,000			&	600		\\
Equatorial surface magnetic field ($\si{nT}$)														&	30,600					&	430,000		&	21,400	\\
Dipole stand-off distance $R_\mathrm{CF}$	($\si{R_\mathrm{P}}$) 				&	$\SI{10}{R_\mathrm{E}}$ & $\SI{46}{R_\mathrm{J}}$ & $\SI{20}{R_\mathrm{S}}$ \\
Observed stand-off distance $R_\mathrm{MP}$ ($\si{R_\mathrm{P}}$)			&	$8-\SI{12}{R_\mathrm{E}}$ & $63-\SI{92}{R_\mathrm{J}}$ & $22-\SI{27}{R_\mathrm{S}}$ \\
\hline
\end{tabular}
\end{table}

This pressure-balance relationship is reflected in the `dipole stand-off distance' $R_\mathrm{CF}$ for each planet given in Table~\ref{intro:table:magnetospherecomparison}, where CF stands for Chapman-Ferraro. This distance, as derived by \citet{chapman1930}, is a theoretical radius of the magnetopause, measured from the planet centre to the sub-solar point on the magnetopause surface, that would be expected if the external solar wind dynamic pressure was balanced exactly by the magnetic pressure associated only with the internal \textit{dipole} magnetic field. We can see that it is much greater for Jupiter and Saturn than for Earth, as expected from the pressure-balance explanation just given.

However at Saturn and Jupiter in particular, the observed magnetopause stand-off distances are significantly larger even than the Chapman-Ferraro estimates. This is mainly due to the significant internal plasma sources at each planet. At Saturn, the icy moon Enceladus orbits at a distance of $\SI{3.95}{R_\mathrm{S}}$, and ejects water group molecules into the magnetosphere at around \SI{150}{kg s^{-1}}, while at Jupiter, the volcanic moon Io orbits at $\SI{5.9}{R_\mathrm{J}}$ and ejects sulphur dioxide into the magnetosphere at around \SI{1000}{kg s^{-1}}~\citep{bagenal2011}. At each planet, this material is partially ionised, and the plasma pressure associated with this population adds to the internal magnetic pressure, inflating the magnetosphere beyond a dipolar internal field model. This is discussed in more detail in Section~\ref{intro:sec:pbalance}. In contrast at Earth, the rocky Moon does not contribute a significant plasma population, and also mainly orbits the planet beyond the magnetosphere boundary \citep[e.g.][]{schneider1967}, and so does not have the same influence o Earth's magnetosphere.

For Jupiter and Saturn, these internal plasma populations not only influence the overall size of the magnetosphere but also the internal structure. This is due to the rapid rotation rates of the two planets, shown by the short spin periods in Table~\ref{intro:table:magnetospherecomparison}. Due to the aforementioned frozen-in field theorem, the ionised plasma is azimuthally accelerated towards co-rotation with the rapidly rotating magnetic field of the magnetosphere. The centrifugal force associated with this confines the plasma towards the rotational equator, creating a thick plasma sheet. In order to balance this centrifugal force, the magnetic field is distorted around the plasma sheet from a dipolar magnetic field into a disc-like `magnetodisc' structure, even on the dayside, with a strong associated magnetic curvature force. This is characterised by field lines that are stretched radially outwards near the equatorial plane in outer magnetosphere, as can be seen particularly for Jupiter in Figure~\ref{intro:fig:magnetospherecomparison}, and is supported by an azimuthal ring current. The intensity of the ring current is enhanced by a population of hotter, more variable plasma that originates in the outer magnetosphere at both Saturn~\citep[e.g.][]{sergis2010} and Jupiter~\citep[e.g.][]{mauk2004}, with observed plasma $\beta$ of the order $2-5$ and ${\sim}100$ for these populations respectively. The magnetic field strength of this disk-like magnetic field structure in general varies more slowly with radial distance than a dipolar magnetic field, and thus also influences pressure balance at the magnetopause. This is investigated for both Saturn and Jupiter in Chapter~\ref{chap:compress}. 

\subsection{Magnetospheric Dynamics}
The simplest dynamical process that a magnetosphere undergoes is compression and expansion under varying solar wind conditions. For example, as the solar wind dynamic pressure increases, due to an increase in velocity or number density by some process as described in Section~\ref{intro:sec:solarwind}, the magnetosphere is compressed. This compression causes the internal magnetic field pressure to increase, until it balances the enhanced external solar wind dynamic pressure and a new equilibrium magnetopause location is reached. As the solar wind pressure decreases, the magnetosphere then inflates. The magnetopause is therefore in constant motion, with a velocity or order $\SI{10}{kms^{-1}}$ at Earth \citep{berchem1982} and $\SI{100}{kms^{-1}}$ at Saturn \citep{masters2011}. The exact response of the magnetosphere to varying $D_\mathrm{P}$ varies significantly between planets due to the different internal structures as discussed in the previous section. For Saturn in particular, the distance that the magnetopause location shifts for a given change in $D_\mathrm{P}$, and how this varies with different internal and external conditions, is the subject of the study in Chapter~\ref{chap:compress}, and is discussed in detail there.

However even under approximately constant solar wind conditions, a magnetosphere is not a static object. At the`solar-wind driven' magnetospheres Earth and Mercury, the key large-scale dynamic process is known as the \textit{Dungey Cycle}, after \citet{dungey1961}. A diagram of this process is shown in Figure~\ref{intro:fig:dungeycycle}. For an IMF oriented anti-parallel to the planetary magnetic field, conditions are favourable for reconnection of the two magnetic fields at the dayside magnetopause. The planetary magnetic field lines are then open to the solar wind at one end whilst still anchored to the planet at the other, and are thus convected downstream to the nightside by the solar wind flow. This leads to a build-up of magnetic flux on open field lines in the magnetotail, which then reconnect across the tail current sheet. This causes plasma to return on closed field lines back towards the planet on the nightside. At Earth, this process is associated with the generation of aurora along the boundary between open and closed magnetic field lines, in the northern and southern polar atmosphere \citep[e.g.][]{milan2007}. As the energetic electrons gyrate around the polar magnetic field lines towards the planet, they excite atoms in the atmosphere, which then decay back to a lower energy level and emit photons in the process, which are observed as aurora.

\begin{figure}
\centering
\noindent\includegraphics[width=0.8\textwidth]{intro/dungeycycle.png}
\caption[Diagram of the Dungey cycle.]{Diagram showing the Dungey Cycle process at Earth, from \citet{dungey1961}. Thin black `lines of force' are magnetic field lines. The solar wind flows from left to right.}
\label{intro:fig:dungeycycle}
\end{figure}

At Jupiter and Saturn, the key dynamical process is driven by the rapid planetary rotation, and thus we say they are `rotationally driven' magnetospheres. This process is known as the Vasyliunas cycle, after \citet{vasyliunas1983}, and a diagram depicting it is shown in Figure~\ref{intro:fig:vasyliunascycle}. As discussed in the previous section, these rapidly rotating magnetospheres have significant internal plasma populations, which are accelerated to near-corotation with the rotating planetary magnetic field. The centrifugal interchange instability transports this plasma radially outwards, such that initially inner cold, dense flux tubes are exchanged with outer hot, tenuous flux tubes \citep{southwood1989}, and the magnetic field is stretched out as shown in region 1 of Figure~\ref{intro:fig:vasyliunascycle}. As the flux tubes rotate around to the nightside, they are no longer as confined by the magnetopause boundary and so expand down the tail, and the magnetic field becomes increasingly more stretched (region 2) until reconnection occurs (region 3). This generates the release of a `plasmoid' down the tail, and the newly empty flux tube is then convected back around the planet on the dawn side. This cycle is dominant over the Dungey cycle for the outer giant planets due to the combined effects of the much faster rotation rates, and overall larger magnetospheric sizes meaning a much longer period of time for a magnetic field line to be convected across the polar cap from dayside to nightside \citep{forsyth2010}. However at Saturn in particular, it still uncertain how much of a role the Dungey cycle has to play \citep[e.g.][]{cowley2005}.

\begin{figure}
\centering
\noindent\includegraphics[width=0.9\textwidth]{intro/vasyliunascycle.png}
\caption[Diagram of the Vasyliunas cycle.]{Diagram showing the Vasyliunas cycle of plasma transport, from \citet{vasyliunas1983}. The dotted line shows the magnetopause boundary, the dashed line shows the region where plasma perfectly corotates with the magnetic field, and empty arrows show the plasma flow direction.}
\label{intro:fig:vasyliunascycle}
\end{figure}

In addition to these main modes of plasma transport in planetary magnetospheres, there are also numerous small-scale dynamical processes, such as Kelvin-Helmholtz vortices on the magnetopause surface; however these vary significantly between planets and are not relevant to the work of this thesis, and so we do not  cover them explicitly here.

\section{Saturn and its Magnetosphere}\label{intro:sec:saturn}
%\subsection{Configuration of Saturn's Magnetosphere}
Saturn orbits the Sun once every ${\sim}29$ years, on an elliptical orbit at an average distance of ${\sim}\SI{9.5}{AU}$. Saturn is approximately 10 times the size of Earth (by radius), and around 100 times as massive, meaning its density is around 1/10 Earth's. This is because Saturn is a `gas giant' planet, composed mainly of molecular hydrogen and helium, with a small rocky core. Between the core and the outer layers, the hydrogen is compressed to such high pressures and temperatures that it becomes `metallic', flowing and conducting electricity and generating the dynamo of Saturn's planetary magnetic field.

The internal planetary magnetic field is approximately dipolar, though with smaller higher order moments, and is offset northwards from the planetary equator by ${\sim}\SI{0.05}{R_S}$ \citep{dougherty2018}. Arguably the most interesting aspect of Saturn's internal magnetic field is that the dipole axis is extremely closely aligned with the planet spin axis, with the same study by \citet{dougherty2018} finding an upper limit of $\SI{0.01}{\degree}$ difference. This makes Saturn unique in the solar system, and it is also seemingly in contradiction with Cowling's Theorem, which states that an active dynamo cannot maintain a perfectly axisymmetric magnetic field \citep{cowling1933}. This extreme axisymmetry also means it is near impossible to determine the rotation rate of the planet's deep interior; however, a range of periodic phenomena are observed at Saturn at periods assumed \textit{close} to the true planetary rotation rate, as discussed in detail in Section~\ref{intro:sec:periodicities}.

From Table~\ref{intro:table:magnetospherecomparison} and the associated discussion, we have seen that in many ways Saturn's magnetosphere is an intermediate between Earth and Jupiter. It is therefore a particularly interesting system to study, and can be used to learn more about magnetospheric physics in a global context. Saturn's magnetosphere was first investigated \textit{in situ} with single flybys by the outer solar system space missions \textit{Pioneer} (1979), \textit{Voyager I} (1980) and \textit{Voyager II} (1981). These observations did not reveal a significant `magnetodisc' magnetic field structure on Saturn's dayside (as was known to exist at Jupiter), although did provide some evidence for a thin current sheet on the dawn flank \citep{smith1980}. However with the arrival of the \textit{Cassini} space mission into orbit around Saturn in 2004, and its continued observation of the system until late 2017, our scientific understanding of Saturn's magnetosphere has been revolutionised. (The \textit{Cassini} space mission is discussed in detail in Chapter~\ref{chap:cassini}.)

We now know that in the outer magnetosphere, a non-negligible magnetodisc magnetic field structure exists at all local times, due to the reasoning discussed in Section~\ref{intro:sec:comparativemagnetospheres}. However particularly on the dayside, it was found that a significant magnetodisc structure only forms under low solar wind dynamic pressure, where the subsolar magnetopause stand-off distance becomes greater than $\SI{23}{R_S}$, and that when it is more extremely compressed than this, the magnetic field remains approximately dipolar~\citep{arridge2008}. It is interesting that this value falls between the two modes of the bimodal distribution in magnetopause stand-off distances observed by \citet{pilkington2015}, of $\SI{20.7}{R_S}$ and $\SI{27.1}{R_S}$. This behaviour is broadly due to overall force balance in Saturn's magnetosphere; for a more expanded system, the magnetic field strength is weaker in the outer magnetosphere, and so a larger magnetic tension force is needed to balance the centrifugal and plasma pressure gradient forces acting radially outwards on the plasma. In Chapter~\ref{chap:compressibility} of this thesis, I present results that show the compressibility of the magnetosphere also changes behaviour at around this value of stand-off distance, at ${\sim}\SI{25}{R_S}$. 

The formation of a magnetodisc structure at Saturn is also influenced by the variable hot plasma population observed by \citet{sergis2010} and others, as discussed in Section~\ref{intro:sec:comparativemagnetospheres}, through an enhancement of the azimuthal ring current intensity. By consideration of Amp\`ere's law, it can be understood that the magnetic field associated with an azimuthal ring current flowing in the direction of corotation decreases Saturn's planetary magnetic field in the inner magnetosphere, and increases it in the outer magnetosphere. This is a magnetodisc magnetic field structure. Many studies have attempted to characterise the thickness and radial extent of this azimuthal ring current, using a combination of \textit{Cassini} data, and models such as that of \citet{connerney1981b, connerney1983}, which assumes an azimuthally symmetric disk of current of uniform thickness. The nature of the ring current has been observed to vary significantly over time, with location, and with system size \citep[e.g.][]{bunce2007}, with a variable thickness of average ${\sim}\SI{3}{R_S}$ and significant radial extent from around $\SI{7}{R_S}$ out to the magnetopause boundary  \citep[e.g.][]{kellett2009}. However due to incomplete coverage particularly across local time for most of the \textit{Cassini} mission, it was not until near the end of the mission that the local time asymmetry of the ring current was demonstrated by \citet{sergis2017}. Indeed, the local time variation in the large scale structure of Saturn's magnetosphere is still not fully understood, and is studied in this thesis in Chapter~\ref{chap:LTsectors} using a flexible model of the ring current adapted from \citet{achilleos2010a}.

\subsection{Planetary Period Periodicities}\label{intro:sec:periodicities}
Dynamic current sheet. A summary of periodicities, the history, the different values, when and where they were discovered and what they might mean.

\subsection{Pressure Balance at the Magnetopause}\label{intro:sec:pbalance}
As previously mentioned, the magnetopause boundary can be approximated as the location where the effective pressure of the solar wind exerted on the magnetosphere is exactly balanced by the sum of the internal magnetospheric particle and field pressures. 

Before impacting on the magnetopause, the solar wind flow is first decelerated via the bow shock, and is deflected around the magnetosphere obstacle. This acts to reduce the dynamic pressure incident on the magnetopause surface, and must be accounted for when considering pressure balance. \citet{petrinec1997} used Bernoulli's equation in combination with the Rankine-Hugoniot jump conditions across the bow shock, assuming adiabatic flow of the solar wind, to show that the relation
\begin{equation}\label{intro:eq:pbalance1}
\frac{B_{\mathrm{MS}}^2}{2\mu_0} + P_{\mathrm{MS}} = kD_\mathrm{P}\cos^2\Psi + P_0\sin^2\Psi
\end{equation}
provides an approximation that is valid across the magnetopause surface, not just at the nose. The subscript {\sc{$\mathrm{MS}$}} denotes magnetospheric properties, such that the terms on the left hand side of equation \ref{intro:eq:pbalance1} are the magnetospheric magnetic and plasma pressures respectively. $D_\mathrm{P}$ is the solar wind dynamic pressure.  $\Psi$ is the flaring angle measured between the upstream flow velocity vector and the normal to the magnetopause surface, such that $\Psi=0$ at the nose and generally increases as you move anti-sunward along the magnetopause surface. The first term on the right hand side is associated with the solar wind dynamic pressure, where $k$ is a positive constant $\leq1$ to account for the aforementioned diversion of flow, and the $\cos^2\Psi$ factor accounts for the reduction in the normal component of dynamic pressure on the flanks and tail of the magnetosphere. The second term on the right hand side is composed of a `static' pressure $P_0$ associated with the thermal pressure of the solar wind, and a $\sin^2\Psi$ factor to ensure a real (i.e. not imaginary) flow velocity in the subsolar region \citet[see][]{petrinec1997}.

In order to improve agreement with the results of MHD simulations from \citet{hansen2005}, and to improve the consistency of $D_\mathrm{P}$ estimates, \citet{kanani2010} proposed a modification to this relation such that $P_0$ is dependent on $D_\mathrm{P}$. The relationship then becomes

\begin{equation}\label{intro:eq:pbalance2}
\frac{B_\mathrm{MS}^2}{2\mu_0} + P_\mathrm{MS} = [k\cos^2(\Psi) + \frac{k_\mathrm{B}T_\mathrm{SW}}{1.16m_\mathrm{p}u_\mathrm{SW}^2}\sin^2(\Psi)] D_\mathrm{P}
\end{equation}

where $k_\mathrm{B}$ is the Boltzmann constant,  $m_\mathrm{p}$ is the mass of a proton, and $T_\mathrm{SW}$ and $u_\mathrm{SW}$ are the solar wind temperature and velocity respectively. The value of $k$ depends on the ratio of specific heats $\gamma$ in the solar wind, and the upstream sonic Mach number $M$. For high (${\gtrsim}8$) Mach number flow with $\gamma = 5/3$, $k = 0.881$ \citep{spreiter1966}, which is a valid assumption for the solar wind at Saturn's orbit \cite[e.g.][]{slavin1985,achilleos2006}.

This relationship allows an approximation of the solar wind dynamic pressure to be made, when only internal information about the state of the magnetosphere is known. 
Thus this relation is often used in studies that attempt to model the shape and size of the magnetopause boundary in response to changing $D_\mathrm{P}$, using only \textit{in situ} \textit{Cassini} data about the magnetic and plasma pressure inside the magnetosphere. %These studies are discussed in more detail in Chapter~\ref{chap:compress}.

\section{A Force-Balance Model of Saturn's Magnetodisc}
Throughout this thesis I will employ the magnetodisc model from \citet{achilleos2010a}, with appropriate modifications as described in each chapter. This model is based on  a magnetic field and plasma model originally constructed for the Jovian magnetodisc by \citet{caudal1986}, and adapted for the Saturn system. More information can be found in \citet{achilleos2010a, achilleos2010b}. The model is axisymmetric about the planetary dipole/rotation axis, which are assumed to be parallel. It is constructed based on the assumption of force balance in the rotating plasma of the magnetosphere between the Lorentz body force (magnetic pressure and tension forces), pressure gradient force and centrifugal force, such that 
\begin{equation}\label{intro:eq:forcebalance}
\boldsymbol{J} \times \boldsymbol{B} = \nabla P - nm_i\omega^2\rho\boldsymbol{e}_\rho
\end{equation}
where $\boldsymbol{J}$ is the current density, $\boldsymbol{B}$ is the magnetic field vector and $\rho$ is cylindrical radial distance from the axis, with $\boldsymbol{e}_\rho$ its unit vector. The plasma properties are isotropic pressure $P$, temperature $T$, ion number density $n$, mean ion mass $m_i$ and angular velocity $\omega$.

By representing the azimuthally-symmetric magnetic field as the gradient of a magnetic Euler potential $\alpha$, \citet{caudal1986} demonstrated that equation~\ref{intro:eq:forcebalance} is equivalent to the partial differential equation
\begin{equation}\label{intro:eq:pde}
\frac{\partial^2\alpha}{\partial r^2} + \frac{1-\mu^2}{r^2} \frac{\partial^2\alpha}{\partial \mu^2} = -g(r,\mu,\alpha)
\end{equation}
where $r$ is radial distance, $\mu = \cos\theta$, the cosine of colatitude, and $g(r,\mu,\alpha)$ is a source function determined by the distribution of plasma and angular velocity in $r,\mu$ space. This equation can be solved semi-analytically using Jacobi polynomials as laid out in detail in \citet[Appendix]{achilleos2010a} to give surfaces of constant $\alpha$, corresponding to magnetic field lines, in $r, \mu$ space. The model solution also provides a prediction of the local plasma pressure, and the azimuthal current density components associated with each of the terms on the right hand side of equation \ref{intro:eq:forcebalance}. 

Since the source function is itself dependent on $\alpha$, equation \ref{intro:eq:pde} must be solved iteratively, starting from a pure dipole magnetic field and then successively perturbing it. At each iteration, a linear combination of the present solution $\alpha_i$ and the previous solution $\alpha_{i-1}$ is used as input for the next iteration calculation, such that
\begin{equation}
\alpha_{i+1\mathrm{(input)}} = \gamma\alpha_i + (1-\gamma)\alpha_{i-1},
\end{equation}
where $\gamma<1$ controls the relative weighting between the previous and current solutions. This is a form of numerical `relaxation'. This $\alpha_{i+1(\mathrm{input})}$ is then used to calculate the source function in equation~\ref{intro:eq:pde} to solve for $\alpha_{i+1}$. In the original model construction, the two components were weighted equally ($\gamma=0.5$) and calculations continued until the maximum difference between successive iterations fell below a chosen `tolerance' $\delta = 0.5\%$, considered as convergence. In some of the work in this thesis I found that, for models with more extreme input parameters, it was necessary to weight the previous solution up to four times more heavily than the present solution ($\gamma=0.2$), in order to achieve convergence. (Exactly what constitutes an `extreme parameter' will be discussed in future chapters.) In order to keep the ratio $\delta/\gamma$ constant at $10^{-2}$, and therefore consistent with the original model approach, this corresponds to using a more stringent stopping tolerance $\delta = 10^{-2}\times0.2 = 0.2\%$  in such cases.

The global plasma properties can then be inferred entirely from the calculated magnetic field structure, using appropriate boundary conditions, as follows. \citet{caudal1986} explained that as a consequence of equation~\ref{intro:eq:forcebalance}, with $T$ and $\omega$ constant along magnetic field lines (according to Liouville's theorem and Ferraro's isorotation theorem respectively), the plasma pressure P is determined by 
\begin{equation}\label{intro:eq:p}
P = P_{0}\exp\left(\frac{\rho^2-\rho_0^2}{2\ell^2}\right),
\end{equation}
where $\ell$ is the `confinement scalelength' (in $\rho$)
\begin{equation}
\ell^2 = \frac{2k_BT}{m_i\omega^2}.
\end{equation}
The subscript 0 means the quantity evaluated at the equatorial crossing point of the magnetic field line. This represents the plasma being confined towards the rotational equatorial plane due to the centrifugal force exerted on it. The model assumes that the plasma is composed of a cold and hot population; for the hot plasma population, where the thermal energy associated with the plasma is significantly greater than the centrifugal potential, $\ell^2$ tends to infinity, such that the hot plasma pressure is not confined to the equator but is constant along magnetic field lines, $P_\mathrm{h} = P_\mathrm{h0}$. This is supported by observations made using data from \textit{Cassini} MIMI, such as \citet{krimigis2007}. Hence, the full form of equation~\ref{intro:eq:p} is only necessary for calculating the cold plasma pressure.

The requisite boundary conditions for the model are, then, the equatorial radial profiles of plasma properties. These were obtained from studies using results mainly from \textit{Cassini} plasma instruments CAPS and MIMI/INCA, as summarized in \citet{achilleos2010a}, and updated in \citet{achilleos2010b}. For the cold plasma population, the profiles for $m_i$ and $T$ were obtained from \citet{wilson2008}, $\omega$ profiles from \citet{wilson2008} and \citet{kane2008}, and the flux tube content information from \citet{mcandrews2009}. For the studies described in Chapters \ref{chap:equinox} and \ref{chap:LTsectors} I updated some of these boundary conditions using more recent results from \citet{wilson2017}, as described in those chapters.

As the hot plasma pressure is assumed uniform along magnetic field lines, the plasma population may be completely characterised by a particular equatorial plasma pressure $P_\mathrm{h0}$ and flux tube volume $V$ per unit of magnetic flux, where
\begin{equation}\label{intro:eq:ftv}
V = \int_{0}^{s_{B}} ds/B,
\end{equation}
and $ds$ is an element of arc length along the magnetic field line. The integral limits represent measurement along a field line of total length $s_B$ between the southern and northern ionospheric footprints at $\SI{1}{R_S}$. The flux tube volume is therefore dependent on both the shape of magnetic field lines, via $ds$, and the strength of the field, via $B$. Studies using \textit{Cassini} MIMI data such as \citet{sergis2007} found that the equatorial pressure associated with the hot plasma population was highly variable with $\rho$ and over time, as described in Section~\ref{intro:sec:saturn}. In light of these observations, the original \citet{achilleos2010a} model simply parameterised the global hot plasma content by a single`hot plasma index' $K_\mathrm{H}$, where $  K_\mathrm{H}= P_\mathrm{H0}V$ is constant beyond $\SI{8}{R_S}$, and $P_\mathrm{H0}$ decreases linearly to 0 inside that distance. A similar parameterisation, though with different values of the constants, was made in \citet{caudal1986}, who argued that for the Jovian system, under the expected conditions of rapid radial diffusion, the hot plasma would be transported isothermally. In \citet{achilleos2010a} the authors used a value of $K_\mathrm{H} = \SI{2e6}{Pa m T^{-1}}$ to represent `typical' hot plasma content conditions at Saturn, however results presented there suggest $K_\mathrm{H}$ may vary in the range $10^5{\--}10^7~\si{Pa m T^{-1}}$. Parameterising the hot plasma content in this way provides the flexibility to very simply characterize the level of ring current activity in the model, and thus investigate the effect of the varying hot plasma content on magnetospheric structure, and magnetospheric compressibility. This is the basis of the study described in Chapter~\ref{chap:compress}. In Chapter~\ref{chap:LTsectors}, this hot plasma pressure boundary condition is updated to describe different local time sectors, using recent results from \citet{sergis2017}.

Finally, at every iteration, a small uniform southward-directed `shielding field' is added to the magnetic field, in order to approximately account for the magnetic field associated with the magnetopause and magnetotail current sheets. Sketches of these current systems are included in the diagram in Figure~\ref{intro:fig:magnetosphere}, though note they are in the opposite sense for the Saturn system due to the opposite orientation of Saturn's planetary dipole. In \citet{achilleos2010a} the magnitude of this field was chosen by calculating dayside equatorial averages of the empirical field models of \citet{alexeev2005} and \citet{alexeev2006}, and it varied with model magnetodisc radius $R_\mathrm{D}$ \citep[see][Figure 6]{achilleos2010a}.  In particular the component of the shielding field associated with the magnetopause currents was based on a dipole approximation of the magnetospheric magnetic field. In Chapter~\ref{chap:equinox} we update this calculation for a more realistic magnetodisk magnetic field, and in Chapter~\ref{chap:LTsectors} we modify this calculation using local time sector averages of the models of \citet{alexeev2005} and \citet{alexeev2006}, to account for the increased significance of the tail current field compared to the magnetopause current field for nightside local time sectors.

\section[Open Questions about Saturn's Magnetosphere]{Open Questions about Saturn's Magnetosphere: Motivations for this Thesis}
Summary of this PhD.

