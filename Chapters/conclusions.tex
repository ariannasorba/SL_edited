\chapter{Conclusions and Directions for Future Work}
\label{chap:conclusions}
\section{Impact of System Size on Large-Scale Structure and Dynamics}
In this thesis we have studied in detail how the large-scale structure and dynamics of Saturn's magnetosphere varies with system size. In Chapter~\ref{chap:compress}, we used the UCL/AGA model to investigate specifically how the compressibility of the magnetosphere varied with system size, and in Chapter~\ref{chap:LTsectors} we looked at how the large-scale structure varied at different system sizes across a range of local times. The size of the magnetosphere is in turn influenced by the upstream solar wind conditions; in Chapter~\ref{chap:intro} we discussed how the magnetopause location is determined to the first order by pressure balance between the solar wind dynamic pressure $D_\mathrm{P}$ and the internal magnetospheric plasma and magnetic pressures, and how $D_\mathrm{P}$ varies with both the density and speed of the solar wind plasma.

Our modelling results presented in Chapters~\ref{chap:compress} and \ref{chap:LTsectors} suggest that the large-scale structure of Saturn's magnetosphere is more disc-like when the system is expanded, beyond a sub-solar magnetopause radius of ${\sim}{25}{R_S}$. This \textit{magnetodisc} configuration is associated with more radially stretched field lines in the middle magnetosphere near the equatorial plane around a thin equatorial current sheet, and a total magnetic field strength that falls more slowly with radial distance than a dipole magnetic field. If the system is compressed by increasing solar wind dynamic pressure, the magnetospheric magnetic field then reconfigures into a more dipolar structure, with a correspondingly thicker equatorial current sheet. 

Similar results were found in the empirical study by \citet{arridge2008}, based on \textit{Cassini} magnetic field observations. In that study, the authors surveyed \textit{Cassini} magnetometer data for current-sheet-like fields, and examined the distribution of observations that fit a certain set of criteria corresponding to a distorted, stretched magnetic field configuration. They found that a significant magnetodisc structure only formed on the dayside when the system was expanded with a sub-solar radius beyond ${\sim}{23}{R_S}$, and that for smaller systems, a magnetodisc only formed on the nightside and flanks. They suggested that this is due to the near-coincidence of the expected inner boundary of the magnetodisc based on stress balance, and the nominal subsolar magnetopause, meaning the disc is prevented from forming under compressed conditions. This is in agreement with the results of the study by \citet{bunce2008}, based on models of the ring current from \citet{bunce2007}, which found that the magnetic moment of the ring current increases as the system expands, causing a more extreme distortion of the magnetospheric magnetic field in such conditions.  

In a steady-state system, we can consider this reconfiguration from a dipolar to a more disc-like structure  as determined by force balance in the rotating plasma of the magnetosphere, between the centrifugal force, plasma pressure gradient force and `$\boldsymbol{J}\times\boldsymbol{B}$' Lorentz body force. In Section~\ref{intro:sec:plasmaforces} we showed how this Lorentz body force is equivalent to the sum of a magnetic pressure gradient force, and a magnetic tension force. Near the equatorial plane, the main force acting radially inwards is the component of the magnetic tension force perpendicular to the magnetic field, which is proportional to $B^2/r_\mathrm{c}$ (where $r_\mathrm{c}$ is the magnetic field line radius of curvature). As the system expands, the magnetic field strength in the outer magnetosphere falls, and so in order for the magnetic tension force to balance the centrifugal and plasma pressure gradient forces acting radially outwards, the radius of curvature of the magnetic field lines must decrease. This corresponds to a more disc-like magnetic field structure, supported by a thin sheet of azimuthal current that extends far into the outer magnetosphere. 

This becomes particularly important when investigating how Saturn's magnetospheric structure varies across local time, as the nominal magnetodisc is significantly larger on the nightside and flanks of the magnetosphere than on the dayside. In addition, recent results from \citet{pilkington2015b} suggest a small dawn-dusk asymmetry in the typical magnetodisc size. In Chapter~\ref{chap:LTsectors}, we presented results of a modelling study which suggested that this variation in system size did indeed influence the magnetic field structure at different local times, which in turn influences magnetic mapping between the equatorial disc and the ionosphere. We found that in particular the open-closed field line boundary mapped to more equatorial regions of the ionosphere for expanded systems than compressed systems, although this effect was small, and of similar significance to the influence of the hot plasma content, discussed later in this chapter.

In Chapter~\ref{chap:equinox} we used a family of UCL/AGA models calculated at different sizes in the range $45{--}\SI{55}{R_S}$, and organised with Saturnian `longitude',  to simulate a periodic thickening and thinning of Saturn's magnetospheric current sheet in response to the dual rotating magnetic perturbations discussed in Section~\ref{intro:sec:periodicities}. We found that, when combined with a geometrical model of a tilted and rippled current sheet surface, this model characterised well the observed periodic perturbations in the magnetic field measured by \textit{Cassini} in Saturn's outer magnetosphere. In particular we found that the meridional component of  the magnetic field was much better characterised when this perturbation in the magnetodisc size was included. This suggests that our approach of using an oscillating boundary location with a $\SI{10}{R_S}$ range appropriately simulates the corresponding change in magnetodisc structure and current sheet thickness, although with caveats as discussed in Chapter~\ref{chap:equinox}. Evidence for periodic perturbations in  the magnetopause boundary was found in the empirical study by \citet{clarke2010}, and also the MHD modelling study \citet{kivelson2014}.

This reconfiguration of the magnetic field under compressed conditions also influences the compressibility of the dayside magnetosphere in response to changing solar wind dynamic pressure. In Chapter~\ref{chap:compress} we presented results of a modelling study which suggested that the magnetosphere becomes more compressible as it expands, due to the more significant magnetodisc structure under such conditions. As previously mentioned, a more significant magnetodisc structure has a magnetic field profile  with strength varying more slowly with radial distance than a  dipole, due to the magnetic field associated with the azimuthal ring current. This affects the pressure balance between the magnetic pressure and the solar wind dynamic pressure at the magnetopause boundary, and thus the compressibility.

We showed in Chapter~\ref{chap:compress} that this is contrast to the Jupiter system, which shows a compressibility behaviour that is approximately constant across system size. This is because Jupiter ... always disc-like. Even though it is closer to sun to higher Dp, combination of rapid rotation rate and plasma sources and massive B field mean never  compressed into a dipole structure even though core is dipolar. If Saturn were moved into Jupiter's place, Dp would be such that the field would always be dipolar. This means Jupiter's compressibility is self-similar across system size.

\section{Impact of Hot Plasma Population on Large-Scale Structure and Dynamics}
More disc-like, thinner and more intense current sheet. Makes it more compressible.

\section{Impact of Periodic Magnetic Perturbations on Large-Scale Structure and Dynamics}
Cause a modulation of the thickness as well as the location. Variation on short timescales corresponding to beat period? Only way to explain Btheta.

\section{Possible Directions for Future Work}
\subsection{Modifications to the UCL/AGA Magnetodisc Model}
- make more parameters LT dependent
- segment them into  smaller LT sectors once more data  is available to make it properly 3D
- pressure anistropy e.g. Nichols paper

\subsection{Investigations into Periodic Magnetic Perturbations}
- recent collaboration opportunities to continue working on equinox study using a combination of northern and southern periodicities

\subsection{Looking Further Afield}
- more large-scale, with the end of Cassini, more  people moving  to Jupiter. Juno may answer some questions, allow a comparison of the compressibility with more detail. JUICE.
Exoplanet discoveries of magnetospheres.