\chapter{Conclusions and Directions for Future Work}
\label{chap:conclusions}
\section{Impact of System Size on Large-Scale Structure and Dynamics}
In this thesis we have studied in detail how the large-scale structure and dynamics of Saturn's magnetosphere varies with system size. In Chapter~\ref{chap:compress}, we used the UCL/AGA model to investigate specifically how the compressibility of the magnetosphere varied with system size, and in Chapter~\ref{chap:LTsectors} we looked at how the large-scale structure varied at different system sizes across a range of local times. The size of the magnetosphere is in turn influenced by the upstream solar wind conditions; in Chapter~\ref{chap:intro} we discussed how the magnetopause location is determined to the first order by pressure balance between the solar wind dynamic pressure $D_\mathrm{P}$ and the internal magnetospheric plasma and magnetic pressures, and how $D_\mathrm{P}$ varies with both the density and speed of the solar wind plasma.

Our modelling results presented in Chapters~\ref{chap:compress} and \ref{chap:LTsectors} suggest that the large-scale structure of Saturn's magnetosphere is more disc-like when the system is expanded, beyond a sub-solar magnetopause radius of ${\sim}\SI{25}{R_S}$. This \textit{magnetodisc} configuration is associated with more radially stretched field lines in the middle magnetosphere near the equatorial plane around a thin equatorial current sheet, and a total magnetic field strength that falls more slowly with radial distance than a dipole magnetic field. If the system is compressed by increasing solar wind dynamic pressure, the magnetospheric magnetic field then reconfigures into a more dipolar structure, with a correspondingly thicker equatorial current sheet. 

Similar results were found in the empirical study by \citet{arridge2008}, based on \textit{Cassini} magnetic field observations. In that study, the authors surveyed \textit{Cassini} magnetometer data for current-sheet-like fields, and examined the distribution of observations that fit a certain set of criteria corresponding to a distorted, stretched magnetic field configuration. They found that a significant magnetodisc structure only formed on the dayside when the system was expanded with a sub-solar radius beyond ${\sim}{23}{R_S}$, and that for smaller systems, a magnetodisc only formed on the nightside and flanks. They suggested that this is due to the near-coincidence of the expected inner boundary of the magnetodisc based on stress balance, and the nominal subsolar magnetopause, meaning the disc is prevented from forming under compressed conditions. This is in agreement with the results of the study by \citet{bunce2008}, based on models of the ring current from \citet{bunce2007}, which found that the magnetic moment of the ring current increases as the system expands, causing a more extreme distortion of the magnetospheric magnetic field in such conditions. This behaviour was also suggested in the original UCL/AGA model study based on a comparison of two models calculated at different sizes \citet{achilleos1010a}.

In a steady-state system, we can consider this reconfiguration from a dipolar to a more disc-like structure  as determined by force balance in the rotating plasma of the magnetosphere, between the centrifugal force, plasma pressure gradient force and `$\boldsymbol{J}\times\boldsymbol{B}$' Lorentz body force. In Section~\ref{intro:sec:plasmaforces} we showed how this Lorentz body force is equivalent to the sum of a magnetic pressure gradient force, and a magnetic tension force. Near the equatorial plane, the main force acting radially inwards is the component of the magnetic tension force perpendicular to the magnetic field, which is proportional to $B^2/r_\mathrm{c}$ (where $r_\mathrm{c}$ is the magnetic field line radius of curvature). As the system expands, the magnetic field strength in the outer magnetosphere falls, and so in order for the magnetic tension force to balance the centrifugal and plasma pressure gradient forces acting radially outwards, the radius of curvature of the magnetic field lines must decrease. This corresponds to a more disc-like magnetic field structure, supported by a thin sheet of azimuthal current that extends far into the outer magnetosphere. 

This becomes particularly important when investigating how Saturn's magnetospheric structure varies across local time, as the nominal magnetodisc is significantly larger on the nightside and flanks of the magnetosphere than on the dayside. In addition, recent results from \citet{pilkington2015b} suggest a small dawn-dusk asymmetry in the typical magnetodisc size. In Chapter~\ref{chap:LTsectors}, we presented results of a modelling study which suggested that this variation in system size did indeed influence the magnetic field structure at different local times, which in turn influences magnetic mapping between the equatorial disc and the ionosphere. We found that in particular the open-closed field line boundary mapped to more equatorial regions of the ionosphere for expanded systems than compressed systems, which has impacts for interpreting ionospheric phenomena such as aurora, as discussed in that chapter.

In Chapter~\ref{chap:equinox} we used a family of UCL/AGA models calculated at different sizes in the range $45{\--}\SI{55}{R_S}$, and organised with Saturnian `longitude',  to simulate a periodic thickening and thinning of Saturn's magnetospheric current sheet in response to the dual rotating magnetic perturbations discussed in Section~\ref{intro:sec:periodicities}. We found that, when combined with a geometrical model of a tilted and rippled current sheet surface, this model characterised well the observed periodic perturbations in the magnetic field measured by \textit{Cassini} in Saturn's outer magnetosphere. In particular we found that the meridional component of  the magnetic field was much better characterised when this perturbation in the magnetodisc size was included. This suggests that our approach of using an oscillating boundary location with a $\SI{10}{R_S}$ range appropriately simulates the corresponding change in magnetodisc structure and current sheet thickness, although with caveats as discussed in Chapter~\ref{chap:equinox}. Evidence for periodic perturbations in  the magnetopause boundary was found in the empirical study by \citet{clarke2010}, and also the MHD modelling study \citet{kivelson2014}.

This reconfiguration of the magnetic field under compressed conditions also influences the compressibility of the dayside magnetosphere in response to changing solar wind dynamic pressure. In Chapter~\ref{chap:compress} we presented results of a modelling study which suggested that the magnetosphere becomes more easily compressible as it expands, due to the more significant magnetodisc structure under such conditions. As previously mentioned, a more significant magnetodisc structure has a magnetic field profile  with strength varying more slowly with radial distance than a  dipole, due to the magnetic field associated with the azimuthal ring current. This affects the pressure balance between the magnetic pressure and the solar wind dynamic pressure at the magnetopause boundary, and thus the compressibility, making the magnetosphere size more sensitive to a given change in solar wind dynamic pressure. A similar result was found in the aforementioned ring-current modelling study of \citet{bunce2007}, due to similar arguments of the magnetodisc structure becoming more significant for an expanded system.

We showed in Chapter~\ref{chap:compress} that this is contrast to the Jupiter system, which seems to show a magnetospheric compressibility behaviour that is approximately constant across system size. We suggested that this was because unlike at Saturn, Jupiter's magnetosphere has a consistent magnetodisc magnetic field structure across the full range of observed system sizes, even when extremely compressed under high solar wind pressure conditions. Therefore the magnetic field pressure, which dominantly controlled the compressibility in our UCL/AGA modelling results, varies self-similarly with radial distance for the different system sizes. This more stable magnetodisc structure at Jupiter is due to a range of factors, first discussed in Section~\ref{intro:sec:comparativemagnetospheres}. Jupiter has a more significant internal source of plasma than Saturn, originating from the volcanic moon Io, and also rotates more rapidly, with a $\SI{9.9}{\hour}$ period compared to Saturn's  ${\sim}\SI{10.6}{\hour}$. In addition, the overall size-scale of Jupiter's magnetosphere is  greater, with typical magnetopause stand-off distance of ${\sim}\SI{75}{R_J}$, due to the larger internal planetary magnetic field at Jupiter (details in Table~\ref{intro:tab:comparativemagnetospheres}), and this internal plasma population. Therefore the centrifugal force exerted radially outwards on the magnetospheric plasma is greater at Jupiter than at Saturn, causing a more significant distortion of the magnetic field into a magnetodisc structure, supported by an intense ring current \citep[e.g.][]{khurana2004}. As a consequence, Jupiter's magnetosphere has a more stable magnetodisc structure particularly on the dayside when compared to Saturn, despite being significantly closer to the Sun and therefore experiencing higher average solar wind dynamic pressures. In contrast, if Saturn were located at Jupiter's position in the solar system, it would show a uniform compressibility behaviour across system size, specifically corresponding to its compressed, more rigid regime. Therefore the impact of varying system size on the large-scale structure and dynamics of each planetary magnetosphere varies depending on both internal and external influences.

\section{Impact of Hot Plasma Population on Large-Scale Structure and Dynamics}
In this thesis we have also investigated in detail how the large-scale structure and dynamics of Saturn's magnetosphere are influenced by the presence of an internal, global hot plasma population. It is pertinent to study the impact of this population in particular as observational studies have shown that this population varies significantly over short timescales \citep{sergis2011} and in local time \citep{sergis2017}, suggesting that its influence may vary across these domains. This is in contrast to the more stable population of cold plasma originating from Saturn's icy moon Enceladus, which is confined towards the equatorial plane due to the centrifugal force acting on it, and thus consistently contributes to the formation of a magnetodisc structure in a predictable way \citep[e.g.][]{arridge2008}. 

In addition, due to the local and global variability in the hot plasma population, it is difficult to study its impact on magnetodisc structure based on \textit{in situ} measurements alone, as a local measurement of plasma properties does not necessarily reveal information about the large-scale distribution of plasma throughout the magnetosphere. The UCL/AGA model is therefore particularly well suited to use as a tool to investigate this issue, as the hot plasma pressure throughout the magnetosphere can be simply parameterised in the model, in line with the expected range based on \textit{in situ} observations, and the corresponding affect on the magnetodisc structure can be readily investigated. 
%This is in contrast to MHD modelling studies such as \citet{jia2012b}, which do not account for a suprathermal plasma pressure population in their model construction.

In Chapters~\ref{chap:compress} and \ref{chap:LTsectors}, we presented results which suggest that in general a globally enhanced hot plasma pressure causes a more disc-like magnetic field structure, with more radially stretched magnetic field lines in the middle magnetosphere. A similar result was found in \citet{achilleos2010b}, who compared two UCL/AGA models calculated with different hot plasma states. This reconfiguration is due to the enhancement of the equatorial ring current intensity associated with a greater hot plasma population, and the associated magnetic field causing a more significant distortion from a dipole in such conditions. This ring current intensity enhancement, and an associated enhancement in equatorial magnetic field strength, is demonstrated explicitly in Chapter~\ref{chap:LTsectors}, when comparing magnetic field profiles for UCL/AGA models representing different local time sectors, with different typical hot plasma pressure profiles. It was shown by \citet{sergis2017} that in general the pressure associated with the hot plasma population was enhanced in the dusk and night local time sectors compared to noon and dawn, and that this suggested an asymmetry in the equatorial profiles of magnetic field strength; of particular interest perhaps was a small between dawn and dusk, with higher magnetic field strength in general at dusk which would be unlikely to reveal itself in noisy \textit{in situ} magnetic field observations. This hot plasma pressure asymmetry also caused a more distorted magnetodisc structure with thinner current sheet at the dusk sector under some conditions; however due to the aforementioned dawn-dusk asymmetry also in system size, this effect was comparable and opposite to the more disc-like structure generated at dawn due to the increased system size there.

In terms of dynamics, in Chapter~\ref{chap:compress} we showed that the hot plasma population also influenced magnetospheric compressibility, increasing the sensitivity of the magnetosphere to changing solar wind conditions with comparable influence as a variation in system size. This is partially explained by the aforementioned reconfiguration of the magnetospheric magnetic field in such conditions, with associated magnetic field pressure varying more slowly with radial distance for systems with higher hot plasma content, and therefore being more easily compressible. However we also showed that, in our UCL/AGA model calculations, for a magnetosphere with sufficiently high hot plasma content still in line with observations from \citet{sergis2007}, the plasma $\beta$ associated with the hot plasma population at the boundary surpassed 1. This means that, under such conditions, the behaviour of the hot plasma pressure, and how it varies with radial distance, becomes important in determining pressure balance at the magnetopause boundary, and thus controlling magnetospheric compressibility. We found that the hot plasma pressure typically varied even more slowly with radial distance than the magnetic pressure associated with a magnetodisc field structure, and thus the magnetosphere became more easily compressible under such conditions. A recent empirical study on the size and shape of Saturn's magnetopause boundary by  \citet{pilkington2015}, also found evidence that the magnetospheric compressibility behaviour was influenced by the magnetosphere's internal plasma state. However as they were limited to using single-point \textit{in situ} observations of magnetospheric conditions, they could only investigate the relationship with the local plasma $\beta$ measured at the magnetopause crossing location, rather than the global state of the hot plasma population in the magnetosphere.

\section{Impact of Periodic Magnetic Perturbations on Large-Scale Structure and Dynamics}
Cause a modulation of the thickness as well as the location. Variation on short timescales corresponding to beat period, hence no clear relationship of lambdaB? Only way to explain Btheta.

\section{Possible Directions for Future Work}
\subsection{Modifications to the UCL/AGA Magnetodisc Model}
- make more parameters LT dependent
- segment them into  smaller LT sectors once more data  is available to make it properly 3D
- pressure anistropy e.g. Nichols paper
- can use to investigate field line mapping even more, and distribution of hot plasma like \citet{sergis2018}. Also make a more realistic set of models for  next section.

\subsection{Investigations into Periodic Magnetic Perturbations}
- use a set of models that don't have different Dp, either with a magnetic perturbation or some other internal trigger that corresponds to the physics of hwat is happening. The exact geometry of  how we sample the model can be influenced too, as desribed in that chapter. Stitch models of different local times together and  then perturb  the thickness.
- recent collaboration opportunities to continue working on equinox study using a combination of northern and southern periodicities
- empirical studies of local time variation in current sheet thickness e.g. Ned

\subsection{Looking Further Afield}
- more large-scale, with the end of Cassini, more  people moving  to Jupiter. Juno may answer some questions, allow a comparison of the compressibility with more detail. JUICE.
Exoplanet discoveries of magnetospheres.