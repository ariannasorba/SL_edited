\chapter{Conclusions and Directions for Future Work}
\label{chap:conclusions}
\section{Summary of Main Conclusions}
\subsection{Impact of Magnetospheric System Size on Large-Scale Structure and Dynamics}
In this thesis we have studied in detail how the large-scale structure and dynamics of Saturn's magnetosphere varies with system size. In Chapter~\ref{chap:compress}, we used the UCL/AGA model to investigate specifically how the compressibility of the magnetosphere varied with system size, and in Chapter~\ref{chap:LTsectors} we looked at how the large-scale structure varied at different system sizes across a range of local times. The size of the magnetosphere is in turn influenced by the upstream solar wind conditions; in Chapter~\ref{chap:intro} we discussed how the magnetopause location is determined to the first order by pressure balance between the solar wind dynamic pressure $D_\mathrm{P}$ and the internal magnetospheric plasma and magnetic pressures, and how $D_\mathrm{P}$ varies with both the density and speed of the solar wind plasma.

Our modelling results presented in Chapters~\ref{chap:compress} and \ref{chap:LTsectors} suggest that the large-scale structure of Saturn's magnetosphere is more disc-like when the system is expanded, beyond a sub-solar magnetopause radius of ${\sim}\SI{25}{R_S}$. This \textit{magnetodisc} configuration is associated with more radially stretched field lines in the middle magnetosphere near the equatorial plane around a thin equatorial current sheet, and a total magnetic field strength that falls more slowly with radial distance than a dipole magnetic field. If the system is compressed by increasing solar wind dynamic pressure, the magnetospheric magnetic field then reconfigures into a more dipolar structure, with a correspondingly thicker equatorial current sheet. 

Similar results were found in the empirical study by \citet{arridge2008}, based on \textit{Cassini} magnetic field observations. In that study, the authors surveyed \textit{Cassini} magnetometer data for current-sheet-like fields, and examined the distribution of observations that fit a certain set of criteria corresponding to a distorted, stretched magnetic field configuration. They found that a significant magnetodisc structure only formed on the dayside when the system was expanded with a sub-solar radius beyond ${\sim}{23}{R_S}$, and that for smaller systems, a magnetodisc only formed on the nightside and flanks. They suggested that this is due to the near-coincidence of the expected inner boundary of the magnetodisc based on stress balance, and the nominal subsolar magnetopause, meaning the disc is prevented from forming under compressed conditions. This is in agreement with the results of the study by \citet{bunce2008}, based on models of the ring current from \citet{bunce2007}, which found that the magnetic moment of the ring current increases as the system expands, causing a more extreme distortion of the magnetospheric magnetic field in such conditions. This behaviour was also suggested in the original UCL/AGA model study based on a comparison of two models calculated at different sizes \citet{achilleos2010a}.

In a steady-state system, we can consider this reconfiguration from a dipolar to a more disc-like structure  as determined by force balance in the rotating plasma of the magnetosphere, between the centrifugal force, plasma pressure gradient force and `$\boldsymbol{J}\times\boldsymbol{B}$' Lorentz body force. In Section~\ref{intro:sec:plasmaforces} we showed how this Lorentz body force is equivalent to the sum of a magnetic pressure gradient force, and a magnetic tension force. Near the equatorial plane, the main force acting radially inwards is the component of the magnetic tension force perpendicular to the magnetic field, which is proportional to $B^2/r_\mathrm{c}$ (where $r_\mathrm{c}$ is the magnetic field line radius of curvature). As the system expands, the magnetic field strength in the outer magnetosphere falls, and so in order for the magnetic tension force to balance the centrifugal and plasma pressure gradient forces acting radially outwards, the radius of curvature of the magnetic field lines must decrease. This corresponds to a more disc-like magnetic field structure, supported by a thin sheet of azimuthal current that extends far into the outer magnetosphere. 

This becomes particularly important when investigating how Saturn's magnetospheric structure varies across local time, as the nominal magnetodisc is significantly larger on the nightside and flanks of the magnetosphere than on the dayside. In addition, recent results from \citet{pilkington2015b} suggest a small dawn-dusk asymmetry in the typical magnetodisc size. In Chapter~\ref{chap:LTsectors}, we presented results of a modelling study which suggested that this variation in system size did indeed influence the magnetic field structure at different local times, which in turn influences magnetic mapping between the equatorial disc and the ionosphere. We found that in particular the open-closed field line boundary mapped to more equatorial regions of the ionosphere for expanded systems than compressed systems, which has impacts for interpreting ionospheric phenomena such as aurora, as discussed in that chapter.

In Chapter~\ref{chap:equinox} we used a family of UCL/AGA models calculated at different sizes in the range $45{\--}\SI{55}{R_S}$, and organised with Saturnian `longitude',  to simulate a periodic thickening and thinning of Saturn's magnetospheric current sheet in response to the dual rotating magnetic perturbations discussed in Section~\ref{intro:sec:periodicities}. We found that, when combined with a geometrical model of a tilted and rippled current sheet surface, this model characterised well the observed periodic perturbations in the magnetic field measured by \textit{Cassini} in Saturn's outer magnetosphere. In particular we found that the meridional component of  the magnetic field was much better characterised when this perturbation in the magnetodisc size was included. This suggests that our approach of using an oscillating boundary location with a $\SI{10}{R_S}$ range in disc radius appropriately simulates the corresponding change in magnetodisc structure and current sheet thickness, although with caveats as discussed in Chapter~\ref{chap:equinox}. Evidence for periodic perturbations in  the magnetopause boundary was found in the empirical study by \citet{clarke2010}, and also the MHD modelling study \citet{kivelson2014}. In Chapter~\ref{chap:equinox} we found significant variation in the estimated phase of rotation at which we expected the maximum displacement of the magnetopause boundary (the parameter $\lambda_\mathrm{B}$) between the \textit{Cassini} passes studied. This suggests this relationship varies on short time scales, corresponding  to the phase relationship between the northern and southern hemispheric magnetic perturbations that generate this behaviour.

This reconfiguration of the magnetic field under compressed conditions also influences the compressibility of the dayside magnetosphere in response to changing solar wind dynamic pressure. In Chapter~\ref{chap:compress} we presented results of a modelling study which suggested that the magnetosphere becomes more easily compressible as it expands, due to the more significant magnetodisc structure under such conditions. As previously mentioned, a more significant magnetodisc structure has a magnetic field profile  with strength varying more slowly with radial distance than a  dipole, due to the magnetic field associated with the azimuthal ring current. This affects the pressure balance between the magnetic pressure and the solar wind dynamic pressure at the magnetopause boundary, and thus the compressibility, making the magnetosphere size more sensitive to a given change in solar wind dynamic pressure. A similar result was found in the aforementioned ring-current modelling study of \citet{bunce2007}, due to similar arguments of the magnetodisc structure becoming more significant for an expanded system.

We showed in Chapter~\ref{chap:compress} that this is contrast to the Jupiter system, which seems to show a magnetospheric compressibility behaviour that is approximately constant across system size. We suggested that this was because unlike at Saturn, Jupiter's magnetosphere has a consistent magnetodisc magnetic field structure across the full range of observed system sizes, even when extremely compressed under high solar wind pressure conditions. Therefore the magnetic field pressure, which dominantly controlled the compressibility in our UCL/AGA modelling results, varies self-similarly with radial distance for the different system sizes. This more stable magnetodisc structure at Jupiter is due to a range of factors, first discussed in Section~\ref{intro:sec:comparativemagnetospheres}. Jupiter has a more significant internal source of plasma than Saturn, originating from the volcanic moon Io, and also rotates more rapidly, with a $\SI{9.9}{\hour}$ period compared to Saturn's  ${\sim}\SI{10.6}{\hour}$. In addition, the overall size-scale of Jupiter's magnetosphere is  greater, with typical magnetopause stand-off distance of ${\sim}\SI{75}{R_J}$, due to the larger internal planetary magnetic field at Jupiter (details in Table~\ref{intro:tab:comparativemagnetospheres}), and this internal plasma population. Therefore the centrifugal force exerted radially outwards on the magnetospheric plasma is greater at Jupiter than at Saturn, causing a more significant distortion of the magnetic field into a magnetodisc structure, supported by an intense ring current \citep[e.g.][]{khurana2004}. As a consequence, Jupiter's magnetosphere has a more stable magnetodisc structure particularly on the dayside when compared to Saturn, despite being significantly closer to the Sun and therefore experiencing higher average solar wind dynamic pressures. In contrast, if Saturn were located at Jupiter's position in the solar system, it would show a uniform compressibility behaviour across system size, specifically corresponding to its compressed, more rigid regime. Therefore the impact of varying system size on the large-scale structure and dynamics of each planetary magnetosphere varies depending on both internal and external influences.

\subsection{Impact of Hot Plasma Population on Large-Scale Structure and Dynamics}
In this thesis we have also investigated in detail how the large-scale structure and dynamics of Saturn's magnetosphere are influenced by the presence of an internal, global hot plasma population. It is pertinent to study the impact of this population in particular as observational studies have shown that this population varies significantly over short timescales \citep{sergis2011} and in local time \citep{sergis2017}, suggesting that its influence may vary across these domains. This is in contrast to the more stable population of cold plasma originating from Saturn's icy moon Enceladus, which is confined towards the equatorial plane due to the centrifugal force acting on it, and thus consistently contributes to the formation of a magnetodisc structure in a predictable way \citep[e.g.][]{arridge2008}. 

In addition, due to the local and global variability in the hot plasma population, it is difficult to study its impact on magnetodisc structure based on \textit{in situ} measurements alone, as a local measurement of plasma properties does not necessarily reveal information about the large-scale distribution of plasma throughout the magnetosphere. The UCL/AGA model is therefore particularly well suited to use as a tool to investigate this issue, as the hot plasma pressure throughout the magnetosphere can be simply parameterised in the model, in line with the expected range based on \textit{in situ} observations, and the corresponding affect on the magnetodisc structure can be readily investigated. 
%This is in contrast to MHD modelling studies such as \citet{jia2012b}, which do not account for a suprathermal plasma pressure population in their model construction.

In Chapters~\ref{chap:compress} and \ref{chap:LTsectors}, we presented results which suggest that in general a globally enhanced hot plasma pressure causes a more disc-like magnetic field structure, with more radially stretched magnetic field lines in the middle magnetosphere. A similar result was found in \citet{achilleos2010b}, who compared two UCL/AGA models calculated with different hot plasma states. This reconfiguration is due to the enhancement of the equatorial ring current intensity associated with a greater hot plasma population, and the associated magnetic field causing a more significant distortion from a dipole in such conditions. This ring current intensity enhancement, and an associated enhancement in equatorial magnetic field strength, is demonstrated explicitly in Chapter~\ref{chap:LTsectors}, when comparing magnetic field profiles for UCL/AGA models representing different local time sectors, with different typical hot plasma pressure profiles. It was shown by \citet{sergis2017} that in general the pressure associated with the hot plasma population was enhanced in the dusk and night local time sectors compared to noon and dawn, and that this suggested an asymmetry in the equatorial profiles of magnetic field strength; of particular interest perhaps was a small between dawn and dusk, with higher magnetic field strength in general at dusk which would be unlikely to reveal itself in noisy \textit{in situ} magnetic field observations. This hot plasma pressure asymmetry also caused a more distorted magnetodisc structure with thinner current sheet at the dusk sector under some conditions; however due to the aforementioned dawn-dusk asymmetry also in system size, this effect was comparable and opposite to the more disc-like structure generated at dawn due to the increased system size there.

In terms of dynamics, in Chapter~\ref{chap:compress} we showed that the hot plasma population also influenced magnetospheric compressibility, increasing the sensitivity of the magnetosphere to changing solar wind conditions with comparable influence as a variation in system size. This is partially explained by the aforementioned reconfiguration of the magnetospheric magnetic field in such conditions, with associated magnetic field pressure varying more slowly with radial distance for systems with higher hot plasma content, and therefore being more easily compressible. However we also showed that, in our UCL/AGA model calculations, for a magnetosphere with sufficiently high hot plasma content still in line with observations from \citet{sergis2007}, the plasma $\beta$ associated with the hot plasma population at the boundary surpassed 1. This means that, under such conditions, the behaviour of the hot plasma pressure, and how it varies with radial distance, becomes important in determining pressure balance at the magnetopause boundary, and thus controlling magnetospheric compressibility. We found that the hot plasma pressure typically varied even more slowly with radial distance than the magnetic pressure associated with a magnetodisc field structure, and thus the magnetosphere became more easily compressible under such conditions. A recent empirical study on the size and shape of Saturn's magnetopause boundary by  \citet{pilkington2015}, also found evidence that the magnetospheric compressibility behaviour was influenced by the magnetosphere's internal plasma state. However as they were limited to using single-point \textit{in situ} observations of magnetospheric conditions, they could only investigate the relationship with the local plasma $\beta$ measured at the magnetopause crossing location, rather than the global state of the hot plasma population in the magnetosphere.

\section{Possible Directions for Future Work}
\subsection{Modifications to the UCL/AGA Magnetodisc Model}
The UCL/AGA model from \citet{achilleos2010a} has been a key tool for investigating the large-scale structure of Saturn's magnetosphere in this thesis. In Chapter~\ref{chap:compress}, we made calculations of the UCL/AGA model over a wider range of parameter space than in previous studies \citet{achilleos2010a, achilleos2010b}, varying both the system size and global hot plasma content to more extreme values, which were still in line with observations. This allowed us to explore in more detail how the behaviour of Saturn's magnetosphere varied across this parameter space. In Chapter~\ref{chap:equinox}, we further extended the applicability of the UCL/AGA model, using a family of models at different sizes to represent a magnetosphere with a periodically oscillating magnetopause boundary, and mainly investigating the dusk flank. For this study we needed to make calculations with larger disc radii than in previous studies, of $\SI{45}{R_S}$ and above, and in our research found that a number of modifications were needed to successfully use the UCL/AGA model in this way. In particular we made adaptations to the calculation of the `shielding field' component of the magnetic field, to account for the larger contribution of the magnetopause current magnetic field for large systems, and also the way in which the model approaches convergence, to approach convergence more slowly for larger models. We also updated the equatorial profile of angular velocity used as a boundary condition in the model, as described in Section~\ref{appendix:sec:wilsonfits}. In Chapter~\ref{chap:LTsectors}, we then extended the applicability of the  UCL/AGA model further still, by making adaptations such that it could be used to represent different local time sectors. We again modified the shielding field component calculation, used local-time-sector-specific hot plasma pressure profiles as boundary conditions, and also updated the equatorial profile of ion temperatures used as boundary conditions, described in Section~\ref{appendix:sec:wilsonfits}. In this way, we have shown how the UCL/AGA model can be adapted and used to investigate all sorts of different phenomena at Saturn.

The study in Chapter~\ref{chap:LTsectors} in particular demonstrated the possibility of effectively extending the UCL/AGA model beyond its original 2-D construction. Further to this, the equatorial profiles of other plasma properties used as boundary conditions in the model could also be made local-time-sector-specific, or otherwise updated using recent results based on \textit{Cassini} observations. For example, the survey of CAPS data from \citet{wilson2017}, which was used to update the plasma angular velocity and temperature profiles, also provides binned observations of temperatures and number densities for hydrogen and water group ions, separated into four local time sectors as in our study. The results presented in \citet{wilson2017} do not show a significant local time asymmetry in these properties, and are also somewhat limited by the lack of data in the dawn sector due to irregular sampling by \textit{Cassini}. However these observations could still be used to constrain how the cold plasma population distribution varies with local time, which may have some impact on the variations in magnetodisc structure we saw in Chapter~\ref{chap:LTsectors}. If possible, observations used to inform plasma boundary conditions could be restricted to even more limited local time sectors with 2 or $\SI{4}{\hour}$ widths, in order to investigate with higher resolution how the magnetodisc structure varies across local time. However the aforementioned incomplete sampling of \textit{Cassini} may make this approach difficult in some regions of the magnetosphere. 

One limitation of the UCL/AGA model is that it assumes the plasma pressure is isotropic, as discussed in Section~\ref{intro:sec:forcebalancemodel}.  This means the model does not account for any resultant force on the plasma associated with the anisotropy of the plasma pressures parallel and perpendicular to the curved magnetic field lines. As the azimuthal currents in the UCL/AGA model are calculated based on the assumption of force-balance (see equation~\ref{intro:eq:forcebalance}, the model therefore does not account for any contribution to this current associated with the pressure anisotropy. Empirical studies based on \textit{Cassini} CAPS data such as \citet{sergis2010} and \citet{kellett2011} found that the current associated with pressure anisotropy is significant particularly in the inner magnetosphere inside $10{\--}\SI{12}{R_S}$, where $P_\perp > P_\parallel$, and that this current flows in the opposite direction  to planetary direction, and therefore opposite to main ring current associated with centrifugal force on the plasma in this region. In this thesis we have mainly  used the UCL/AGA model to investigate Saturn's middle and outer magnetosphere beyond this region, and so our results are unlikely to be affected by this assumption of plasma isotropy. However an adaptation to the model to include pressure anisotropy may benefit future studies. A recent study by \citet{nichols2015} used a similar model construction to \citet{achilleos2010a}, also based on \citet{caudal1986}, to investigate Jupiter's magnetodisc, and found that pressure anisotropy currents were significant in Jupiter's middle magnetosphere in particular, and that overall agreed better with \textit{in situ} plasma data than an earlier model \citep{nichols2011} that did not include pressure anisotropy.

An improved UCL/AGA model that can more accurately describe the structure of Saturn's magnetodisc at different local times and under different conditions, could be useful for many future investigations of Saturn's magnetosphere. For example, a model for  mapping magnetic field lines from the equatorial disc to the ionosphere under  different conditions could be useful for investigating ionospheric phenomena such as aurora, and for investigating the relationship between plasma properties at high latitudes and the equator as in \citet{sergis2018}. As an extension to the study  in Chapter~\ref{chap:equinox}, we could construct a  more physically realistic family of models to represent the periodic perturbation of the magnetopause boundary, that would be applicable to other datasets with different ranges in local time.

\subsection{Improved Studies of Periodic Perturbations in Saturn's Current Sheet}
The periodic perturbations in the position and the thickness of Saturn's equatorial current sheet, discussed in Chapter~\ref{chap:equinox}, are still actively being investigated in the community. In particular there is still a great deal of research ongoing into the nature of the rotating magnetic perturbations discussed in Section~\ref{equinox:sec:intro}, associated with hemispheric field-aligned current systems located at ${\sim}\SI{12}{R_S}$, and which appear to cause this periodic behaviour of the current sheet in the outer magnetosphere. In fact, since the study in Chapter~\ref{chap:equinox} was completed, the picture of the two magnetic perturbations being approximated by rotating transverse dipoles, as shown in Figure~\ref{equinox:fig:cowleydiagrams}, has been further developed into a more complicated picture \citep[][Figure 13]{provan2018}. This and other studies, such as \citet{dougherty2018}, have employed the unique and powerful datasets measured during \textit{Cassini's} Grand Finale orbits (discussed in Section~\ref{cassini:sec:timeline}) to learn more about the nature of Saturn's magnetic field. It is therefore a particularly exciting time for this area of research.

There are numerous ways in which the study presented in Chapter~\ref{chap:equinox} could be taken further. As discussed in that chapter, the family of UCL/AGA magnetodisc models used  to simulate the breathing behaviour did not represent a system under constant solar wind dynamic pressure, and also did not account for any local time variation in magnetodisc structure. We have developed our understanding of the second issue  in particular with our study in Chapter~\ref{chap:LTsectors}, and therefore it may be possible in future to construct a more realistic family  of models, or otherwise vary the coordinate system with which the models are sampled as discussed in Section~\ref{equinox:sec:conclusions}, in order to account for a local time variation in current sheet structure. This would allow the combined flapping and breathing model to be applied to other \textit{Cassini} trajectories, where the range in local time may be wider than the relatively small window we investigated in Chapter~\ref{chap:equinox}.

In addition, the flapping and breathing  model did not explicitly account for the phase difference between the northern and southern magnetic perturbations as it was beyond the scope of that study, and instead organised the model with  respect to the southern magnetic perturbation from \citet{andrews2012}. On publication of the study in \citet{sorba2018}, Stan Cowley at the University of Leicester suggested that we could attempt to account for the influence of both the northern and southern perturbations directly by simply summing terms representing the flapping (equation~\ref{equinox:eq:zcs}) and breathing (equation~\ref{equinox:eq:RD}) modes, for both northern and southern perturbations. In this construction, the values for the flapping and breathing parameters could be fixed following theoretical expectations from \citet[e.g.][]{cowley2017a} and discussed in Section~\ref{equinox:sec:intro}, with $\lambda_0 = \SI{180}{\degree}$ for both the northern and southern perturbations, and $\lambda_\mathrm{B}=\SI{0}{\degree}$ for the northern and $\SI{180}{\degree}$ for the southern perturbations. In addition, the amplitude of the northern and southern perturbations were near equal during the period near equinox studied in  Chapter~\ref{chap:equinox}, with further simplifies the construction as the relative amplitudes do not need  to be accounted for when investigating this period specifically. This would significantly reduce the number of parameters  in the combined flapping and breathing model. This would perhaps enable  us to strengthen the conclusions made in Chapter~\ref{chap:equinox}, and also potentially investigate, for example, the behaviour of the wave parameter $D=\Omega_\mathrm{S}/v_\mathrm{W},$, which controls how the information about  the magnetic perturbations propagate out to the outer magnetosphere. In our study we used a fixed value of $D=\SI{3}{\degree \per R_S}$ due to constraints of fitting so many free parameters, but it would be interesting to investigate how this parameter may vary with radial distance and local time, especially as some other studies looking  into these periodic phenomena also use a fixed value of $D$.

\subsection{Looking Further Afield}
- more large-scale, with the end of Cassini, more  people moving  to Jupiter. Juno may answer some questions, allow a comparison of the compressibility with more detail. JUICE.
Exoplanet discoveries of magnetospheres.